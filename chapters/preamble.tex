%---------------------------------------------------------------------------
% Preface

%\chapter*{Preface}

%Blah blah \dots

% \cleardoublepage

%---------------------------------------------------------------------------
% Table of contents

 \setcounter{tocdepth}{2}
 \tableofcontents

 \cleardoublepage

%---------------------------------------------------------------------------
% Abstract

\chapter*{Abstract}
 \addcontentsline{toc}{chapter}{Abstract}
    This thesis aims to reduce the deviation between calculated and measured heating demands and find the short-comings in SIA 380/1 calculation method. A residential building and an office building are accurately modeled and calculated using EnergyPlus and SIA 180 standard. Both buildings are firstly calibrated based on historical annual heating demand and hourly indoor temperature, then several key building parameters are changed into different values. Based on a large number of simulations, the result indicated that the most influential parameters in simulation are key area temperature heating set points, external wall solar absorptance, infiltration and lighting schedule. In order to reduce the performance gap, it is recommended to create an accurate building envelope and apply accurate outdoor environment. Therefore, an update to SIA standard weather data is also recommended \cite{FREI2017421}.


 \cleardoublepage

%---------------------------------------------------------------------------
% Symbols
\begin{comment}
\chapter*{Nomenclature}\label{chap:symbole}
 \addcontentsline{toc}{chapter}{Nomenclature}

\section*{Symbols}
\begin{tabbing}
 \hspace*{1.6cm} \= \hspace*{8cm} \= \kill
 $\mathrm{EHC}$ \> Conditional equation \> [$-$] \\[0.5ex]
 $e$ \> Willans coefficient \> [$-$] \\[0.5ex]
 $F,G$ \> Parts of the system equation \> [\unitfrac[]{K}{s}]
\end{tabbing}

\section*{Indicies}
\begin{tabbing}
 \hspace*{1.6cm}  \= \kill
 a \> Ambient \\[0.5ex]
 air \> Air
\end{tabbing}

\section*{Acronyms and Abbreviations}
\begin{tabbing}
 \hspace*{1.6cm}  \= \kill
 NEDC \> New European Driving Cycle \\[0.5ex]
 ETH \> Eidgen\"{o}ssische Technische Hochschule
\end{tabbing}

 \cleardoublepage
\end{comment}
%---------------------------------------------------------------------------
