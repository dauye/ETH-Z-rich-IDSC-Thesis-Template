\chapter{Discussion}
    From the calculation and simulation results above, there are a few points that worth mentioning. For static calculation, the SIA 380/1 standard calculatiom method can achieve a reasonably accurate result when an updated weather information is implemented. The standard can probably consider using internal (net) area instead of external (gross) area for energy calculation, as it provides a more accurate results when net area (excluding walls) is used in calculation.\\
    
    For dynamic simulation, weather information is also critical as global warming effect and heat island effect makes the outdoor temperature greatly differ from its assumed values. Also, a more accurate algorithm to simulate heat island effect is needed, as the temperature in city area can differ largely from the weather station's temperature even they are very close to each other. In addition, calibration process is important for analyzing the hourly building performance and gives more confident in annual energy analyses.\\
    
    At the current stage, it is known that the performance gap come from a number of wrong assumptions of building parameters and schedules. However, these wrong assumptions managed to achieve an internal balance and the combination of these current standard assumptions can indeed provide a reasonably accurate results in macroscopic scale (annual per square meter demand for example). For example, an over estimated heat convection coefficient may lead to a higher energy demand, but an over estimated internal loads would lead to a lower energy demand, and the effect of both over estimation cancel each other. To obtain true accuracy, review or modification are needed for not only a single parameters but also all the others. Improving a single assumption without considering others would even lead to more diverse and inaccurate results.\\