\chapter{Conclusion}
    In conclusion, this thesis aims to reduce the deviation between calculated and measured heating demands and find the limitations and short-comings in SIA 380/1 calculation method. A residential building and an office building are accurately modeled and calculated using EnergyPlus and SIA 180 standard calculation method. Both buildings are calibrated then subjected to a large number of simulations with different parameters. The results indicate that the most influential parameters in simulation are key area temperature heating set points, external wall solar absorptance, infiltration and key area lighting schedules. The results also show that the current SIA 380/1 and SIA 2024 schedule would provide a reasonably accurate result if correct weather information is obtained. This is, however, due to a mutuel cancelation effect of a few coupling unappropriated assumptions. In order to obtain true simulation accuracy and reduce the performance gap, it is recommended to create an accurate building envelope and apply accurate outdoor environment, as well as using a set of more comprehensive and accurate occupancy and schedule assumptions.
